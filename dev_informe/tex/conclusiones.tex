\section{Conclusiones}

Pudimos observar y verificar lo siguiente:

\begin{itemize}

\item Utilizando un algoritmo greedy, pudimos resolver nuestro problema buscando iterativamente optimos locales, hasta finalmente alcanzar un optimo global: que Sophia gane el juego.
\item Tras realizar un analisis de complejidad, tanto haciendo un seguimiento del algoritmo planteado, como realizando pruebas empiricas, se concluyo que el algoritmo es de orden lineal: O(n).
\item Al realizar pruebas, tanto las otorgadas por la catedra como las nuestras, se pudo llegar a la conclusion de que no importa la variabilidad de los valores de las monedas, siempre se llega al optimo global.

\end {itemize}

En conclusion, el presente trabajo permitio afianzar los conocimientos adquiridos en la materia de una manera practica, donde desarrollamos un algoritmo greedy para resolver el problema planteado, con una complejidad lineal, alcanzando de esta manera su optimo global (Sophia siempre gana el juego). 