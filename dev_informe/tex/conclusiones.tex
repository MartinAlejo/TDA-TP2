\section{Conclusiones}

Pudimos observar y verificar lo siguiente:

\begin{itemize}

\item Utilizando la tecnica de programacion dinamica con memorizacion, pudimos reducir la complejidad de un algortimo exponencial a uno cuadrático, al no tener que recalcular problemas anteriormente resueltos.
\item Tras realizar un analisis empirico, pudimos confirmar que efectivamente la complejidad de nuestro algoritmo se vio beneficiada por la tecnica de programacion dinamica.
\item la variabilidad de los valores de las monedas \textbf{no} afecta en el tiempo de ejecución del algortimo.
\end {itemize}

En conclusión, el presente trabajo permitió afianzar los conocimientos adquiridos en la materia de una manera práctica, donde desarrollamos un algortimo con programación dinámica para resolver el problema planteado, con una complejidad cuadrática, obteniendo de esta manera la ganancia máxima que pudo haber adquirido Sophia.