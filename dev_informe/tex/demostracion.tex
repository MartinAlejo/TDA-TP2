\section{Demostración del Algortimo}

En un problema que involucra programacion dinamica, para hallar la solución óptima, o sea que nos de el \textbf{máximo valor acumulado posible}, se debe cumplir que:

\begin{enumerate}
    \item La solución óptima de un problema grande puede obtenerse combinando soluciones óptimas de subproblemas más pequeños.
    \item Esta descomposición debe hacerse de tal manera que no se omita ninguna posibilidad relevante para alcanzar la solución óptima.
    \item Se debe tomar una decisión sobre cuál subproblema o combinación de subproblemas proporciona el valor máximo.
\end{enumerate}

Vamos a demostrar entonces que nuestro algoritmo siempre encuentra la maxima ganancia que puede obtener Sophia:

\begin{itemize}
    \item Nuestro caso base es el siguiente: Si no hay monedas, Sophia tendría ganancia cero ya que no habría monedas para agarrar.
    \item Un subproblema consiste en determinar la ganancia máxima que Sophia puede obtener en un intervalo de monedas dado por los índices \(i\) y \(j\) en el arreglo de monedas (donde \(i\) representa el índice de la primera moneda, y \(j\) el de la última). Para cada subproblema, Sophia debe tomar la decisión de sí:
    \begin{itemize}
        \item Agarrar la primera moneda \(m_i\), lo que lleva a resolver el subproblema de las monedas entre los índices \(i+2\) y \(j\) (si Mateo también agarró la primera moneda) o \(i+1\) y \(j-1\) (si Mateo agarró la última moneda).
        \item Agarrar la última moneda \(m_j\), lo que lleva a resolver el subproblema de las monedas entre los índices \(i+1\) y \(j-1\) (si Mateo agarró la primera moneda) o \(i\) y \(j-2\) (si Mateo también agarró la última moneda).
    \end{itemize}
    \item Como se observa en el ítem anterior, nuestros subproblemas se vuelven cada vez más pequeños, hasta eventualmente alcanzar el caso base (enfoque Top Down). A medida que vamos resolviendo estos subproblemas, memorizamos sus resultados para no tener que recalcularlos. Al considerar todas las posibles decisiones que puede tomar Sophia (y las que va a tomar Mateo en consecuencia), hemos considerado (y memorizado) todas las soluciones a los subproblemas posibles, y al combinarlos hallamos la solución óptima para nuestro problema original.
    \item Es decir, dado que cada subproblema se resuelve de forma óptima y se usan esas soluciones para construir la solución final, el algoritmo garantiza que dicha solución será la ganancia máxima que Sophia puede obtener.
\end{itemize}