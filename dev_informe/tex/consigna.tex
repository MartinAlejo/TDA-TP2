\section*{Consigna}


{\Large Introducción y primeros años}
\vskip0.5cm

Cuando Mateo nació, Sophia estaba muy contenta. Finalmente tendría un hermano con quien jugar. Sophi tenía 3 años cuando Mateo nació. Ya desde muy chicos, ella jugaba mucho con su hermano.

Pasaron los años, y fueron cambiando los juegos. Cuando Mateo cumplió 4 años, el padre de ambos le explicó un juego a Sophia: Se dispone una fila de $n$ monedas, de diferentes valores. En cada turno, un jugador debe elegir alguna moneda. Pero no puede elegir cualquiera: sólo puede elegir o bien la primera de la fila, o bien la última. Al elegirla, la remueve de la fila, y le toca luego al otro jugador, quien debe elegir otra moneda siguiendo la misma regla. Siguen agarrando monedas hasta que no quede ninguna. Quien gane será quien tenga el mayor valor acumulado (por sumatoria).

El problema es que Mateo es aún pequeño para entender cómo funciona esto, por lo que Sophia debe elegir las monedas por él. Digamos, Mateo está “jugando”. Aquí surge otro problema: Sophia es muy competitiva. Será buena hermana, pero no se va a dejar ganar (consideremos que tiene 7 nada más). Todo lo contrario. En la primaria aprendió algunas cosas sobre algoritmos greedy, y lo va a aplicar.
\vskip0.5cm
{\Large Consigna}
\vskip0.5cm

\begin{enumerate}
    \item Hacer un análisis del problema, y proponer un algoritmo greedy que obtenga la solución óptima al problema planteado: Dados los $n$ valores de todas las monedas, indicar qué monedas debe ir eligiendo Sophia para sí misma y para Mateo, de tal forma que se asegure de ganar siempre. Considerar que Sophia siempre comienza (para sí misma).
    \vskip0.3cm
    \item Demostrar que el algoritmo planteado obtiene siempre la solución óptima (desestimando el caso de una cantidad par de monedas de mismo valor, en cuyo caso siempre sería empate más allá de la estrategia de Sophia).
    \vskip0.3cm
    \item Escribir el algoritmo planteado. Describir y justificar la complejidad de dicho algoritmo. Analizar si (y cómo) afecta la variabilidad de los valores de las diferentes monedas a los tiempos del algoritmo planteado.
    \vskip0.3cm
    \item Analizar si (y cómo) afecta la variabilidad de los valores de las diferentes monedas a la optimalidad del algoritmo planteado.
    \vskip0.3cm
    \item Realizar ejemplos de ejecución para encontrar soluciones y corroborar lo encontrado. Adicionalmente, el curso proveerá con algunos casos particulares que deben cumplirse su optimalidad también.
    \vskip0.3cm
    \item Hacer mediciones de tiempos para corroborar la complejidad teórica indicada. Agregar los casos de prueba necesarios para dicha corroboración. Esta corroboración empírica debe realizarse confeccionando gráficos correspondientes, y utilizando la técnica de cuadrados mínimos. Para esto, proveemos una explicación detallada, en conjunto de ejemplos.
    \vskip0.3cm
    \item Agregar cualquier conclusión que les parezca relevante.
\end{enumerate}


